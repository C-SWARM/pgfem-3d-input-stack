\documentclass[]{article}
\usepackage{fullpage}
\usepackage{hyperref}
\usepackage{verbatim}
\usepackage{listings}
\usepackage{color}
\usepackage{booktabs}
\usepackage{cprotect}
\linespread{1.1}

\lstset{basicstyle=\ttfamily}

\def\pgf{\emph{PGFem3D}}
\def\con{\emph{con3d++}}
\def\gap{\vspace*{5mm}}

\title{\con{} - Input Stack Generation for \pgf{}}
\author{Matt Mosby}
\date{\today}

\begin{document}
\maketitle
\begin{abstract}
  A user guide for the converter utility \con. The purpose, usage,
  options, and file formats are specified. Known bugs are delineated
  and a brief overview of the internal code structure is provided from
  a programmer's perspective to aid in expanding the utility.
\end{abstract}
\section{Description}
The utility \con{} is used to generate a set of input files for
running the parallel finite element code \pgf. The converter is run
after the finite element mesh has been generated and partitioned using
\emph{t3d2psifel}. The converter sets material properties for the
elements, sets degrees of freedom for nodes, and applies nodal
forces. If specified, \con{} also determines nodal periodicity and
generates cohesive element input files.

\section{Usage}
Running \con{} with no options gives the help message:
\begin{center}
\begin{minipage}{0.8\textwidth}
\begin{lstlisting}
con3d++ -- Create set of input files for PGFem3D.
Usage: con3d++ -np [nproc] -f [base filename] [options]

Options:
-pr args(3):	Periodic unit cell. Args = x,y,z dims
-coh:		Cohesive modeling with interface elements
-v:		Print extra logging info to stderr
-h,--help:	Print this help message
\end{lstlisting}
\end{minipage}
\end{center}

The \verb=-pr= option sets the periodic nodal boundary conditions and
requires an additional input file named
\verb=[base filename].out.periodic=. This file contains the lists of
periodic model entities on each set of periodic edges and faces of the
unit cell. The format of the periodicity input file is specified in
Section \ref{sec:format:per}.

The \verb=-coh= option treats all elements associated with 'interface'
model entities as cohesive elements. The cohesive elements are
filtered and output separately in files named \verb=*.in.co=. The
format of these files is specified in Section \ref{sec:format:coh}.

The \verb=-v= option currently is not implemented. No additional
output is generated.

\section{File Formats}\label{sec:format}
This section describes the input and output file formats used/need by
\con{}. For all file formats, pseudo-code is used to delineate what
information is given by each entry in the file. Comments are
\emph{not} supported in \emph{any} of the files described in this
section. Note that there are different mesh input file formats for
serial versus parallel only. The same output file formats are used for
both parallel and serial simulations using \pgf{}. Please see the
\verb=examples= directory for demonstration of these file formats.
\subsection{\emph{t3d2psifel} output file}\label{sec:format:t3d2psifel}
The use of \emph{t3d2psifel} is not covered in this document, however
\con{} uses its output directly as input and thus the
\emph{t3d2psifel} output file format is described in this section. The
partitioned mesh files must be named \verb=[base filename].out.*=
where \verb=*= denotes the partition number. If \verb=nproc=$=1$, then
the unpartitioned mesh file \verb=[base filename].out= is used. File
is contains the following blocks of information in order
\verb=HEADER=, \verb=NODES=, \verb=ELEMENT_BLOCK=(s).

\gap
\noindent{\bf \verb=HEADER= format}
\begin{center}
  \begin{minipage}{0.9\textwidth}
    \begin{lstlisting}
unused unused elem_degree remaining_line_is_unused
n_nodes n_edges n_trias n_quads n_tetras n_pyrams n_wedges n_hexas
    \end{lstlisting}
  \end{minipage}
\end{center}

\gap
\noindent{\bf \verb=NODES= format (serial)}
\begin{center}
  \begin{minipage}{0.9\textwidth}
    \begin{lstlisting}
id    x_coord y_coord z_coord    model_type model_id model_prop
    \end{lstlisting}
  \end{minipage}
\end{center}

\gap
\noindent{\bf \verb=NODES= format (parallel)}
\begin{center}
  \begin{minipage}{0.9\textwidth}
    \begin{lstlisting}
id gid own  x_coord y_coord z_coord  model_type model_id model_prop
    \end{lstlisting}
  \end{minipage}
\end{center}

\gap
\noindent{\bf \verb=ELEMENT_BLOCK= format}
\begin{center}
  \begin{minipage}{0.9\textwidth}
    \begin{lstlisting}
id connectivity_table model_type model_id model_prop neighbor_info
    \end{lstlisting}
  \end{minipage}
\end{center}
Edge, pyramid, and quadratic elements (except quadratic tetrahedrons)
are not supported by \con{}. The \verb=neighbor_info= field is a
series of integers representing information about the neighboring
elements. For each element type, the number of entries in
\verb=neighbor_info= is given in Table \ref{tab:n_info}.
\begin{table}[htb]
  \centering
  \cprotect\caption{Number of entries in \verb=neighbor_info= for each element type.}
\label{tab:n_info}
\begin{tabular}{rccccc}
\toprule
Format&TRIA&QUAD&TETRA&WEDGE&HEX\\
\midrule
Parallel&3&4&8&10&12\\
Serial&6&8&12&15&18\\
\bottomrule
\end{tabular}
\end{table}

\cprotect\subsection{\verb=*.out.header= file}\label{sec:format:header}
This file defines many aspects of the linear solver, as well as
material properties and their assignment, and assignment of boundary
conditions. The file format is as follows:
\begin{center}
  \begin{minipage}{0.9\textwidth}
\begin{lstlisting}
is_quadratic n_dims n_materials n_concentrations n_orientations

lin_maxit lin_tol epsilon

material_property_0(13 floats then 2 integers)
material_property_1(13 floats then 2 integers)
...
material_property_n(13 floats then 2 integers)

concentration_0 concentration_1 ... concentration_n
orientation_basis_0 ... orientation_basis_n

n_material_assignments
feature_type_0 feature_id_0 material_id_0 fiber_id_0
...
feature_type_n feature_id_n material_id_n fiber_id_n

n_prescribed_features
feature_type_0 feature_id_0 dof0 dof1 dof2
...
feature_type_n feature_id_n dof0 dof1 dof2

n_prescribed_displacements
displacements(n_prescribed_displacements floats)

n_prescribed_forces
feature_type_0 feature_id_0 F_x F_y F_z
...
feature_type_n feature_id_n F_x F_y F_z
\end{lstlisting}
  \end{minipage}
\end{center}
In the file format, the feature types are given by an integer
representation shown in Table \ref{tab:type_map}.
\begin{table}[htb]
\centering
\caption{Feature type identifiers.}\label{tab:type_map}
\begin{tabular}{ccccccc}
\toprule
VERTEX&CURVE&SURFACE&REGION&PATCH&SHELL&INTERFACE\\
\midrule
1&2&3&4&5&6&7\\
\bottomrule
\end{tabular}
\end{table}

\subsection{Periodic input file}\label{sec:format:per}
The periodic file (\verb=*.out.periodic=) defines what model entities
are periodic to each other on a rectangular prism unit cell. The file consists of six lists with the following format:
\begin{center}
  \begin{minipage}{0.9\textwidth}
\begin{lstlisting}
n_entities
type_0 id_0 type_1 id_1 ... type_n id_n
\end{lstlisting}
  \end{minipage}
\end{center}
The lists are associated with \verb=Y_EDGES=, \verb=Z_EDGES=,
\verb=X_EDGES=, followed by \verb=Z_FACES=, \verb=X_FACES=,
\verb=Y_FACES=, and are read in that order. Note that edges have four
entities in each periodic group, whereas faces only have
two. Specifying \verb=n_entities= not equally divisible by the
associated group size is a logical error and will abort \con{}.

\subsection{\con{} main output}\label{sec:format:out}
The main output of the code is a series of files named \verb=*_i.in=,
where \verb=i= is the domain id. Each file contains the mesh
information, boundary conditions and material property assignment for
the domain \verb=i=. The file is formatted as follows:
\gap
\noindent{\bf \verb=HEADER=}
\begin{center}
  \begin{minipage}{0.9\textwidth}
\begin{lstlisting}
n_nodes n_dims n_elements

lin_maxit lin_tol epsilon

n_materials n_concentrations n_orientations

n_elem_nodes_0 n_elem_nodes_1 ... n_elem_nodes_n
\end{lstlisting}
  \end{minipage}
\end{center}

\gap
\noindent{\bf \verb=NODES=}
\begin{center}
  \begin{minipage}{0.9\textwidth}
\begin{lstlisting}
gid own id x_coord y_coord z_coord  model_type model_id model_prop
\end{lstlisting}
  \end{minipage}
\end{center}

\gap
\noindent{\bf \verb=DERICHLET BC=}
\begin{center}
  \begin{minipage}{0.9\textwidth}
\begin{lstlisting}
n_constrained_nodes
node_id_0 dof_x dof_y dof_z
node_id_1 dof_x dof_y dof_z
...
node_id_n dof_x dof_y dof_z

n_displacements
displacement_0 ... displacement_n
\end{lstlisting}
  \end{minipage}
\end{center}

\gap
\noindent{\bf \verb=ELEMENTS=}
\begin{center}
  \begin{minipage}{0.9\textwidth}
\begin{lstlisting}
id connectivity_table model_type model_id model_prop neighbor_info
\end{lstlisting}
  \end{minipage}
\end{center}

\gap
\noindent{\bf \verb=MATERIAL ASSIGNMENT=}
\begin{center}
  \begin{minipage}{0.9\textwidth}
\begin{lstlisting}
element_id material_id fiber_id (typically same as material_id)
\end{lstlisting}
  \end{minipage}
\end{center}

\gap
\noindent{\bf \verb=MATERIAL PROPERTIES=}
Properties, concentrations and orientations are written verbatim as
read from the header file (see Section \ref{sec:format:header}).

\gap
\noindent{\bf \verb=???=}
The meaning of this line is unknown, but is required by \pgf{}.
\begin{center}
  \begin{minipage}{0.9\textwidth}
\begin{lstlisting}
0.5 0.5 0.5
\end{lstlisting}
  \end{minipage}
\end{center}

\gap
\noindent{\bf \verb=NODAL FORCES=}
\begin{center}
  \begin{minipage}{0.9\textwidth}
\begin{lstlisting}
n_loaded_nodes
node_id_0 F_x F_y F_z
node_id_1 F_x F_y F_z
...
node_id_n F_x F_y F_z
\end{lstlisting}
  \end{minipage}
\end{center}

\gap
\noindent{\bf \verb=VOLUME AND SURFACE LOADS=}
The solver does not support volume or surface loads, however they are
specified in the input as
\begin{center}
  \begin{minipage}{0.9\textwidth}
\begin{lstlisting}
0

0
\end{lstlisting}
  \end{minipage}
\end{center}

\subsection{\con{} cohesive output}\label{sec:format:coh}
The cohesive elements are output in files named \verb=*_i.in.co= for
each domain \verb=i=. Note that the cohesive interface properties are
specified separately and are not processed by \con{}. The cohesive
files are formated as follows:
\begin{center}
  \begin{minipage}{0.9\textwidth}
\begin{lstlisting}
0

n_coh_elements
n_coh_elem_nodes_0 n_coh_elem_nodes_1 ... n_coh_elem_nodes_n

elem_id connectivity_table 0 material_id model_prop
...
\end{lstlisting}
  \end{minipage}
\end{center}

\section{Bugs}

\section{Programmer's Guide}
\subsection{General notes}
Most of the containers inherit from \verb=std::list=. Therefore, the
safest way to loop over elements in any container is to use
\verb=iterator=s.

\end{document}